\documentclass[a4paper, 12pt]{article}

\usepackage[ngerman]{babel}
\usepackage[T1]{fontenc}
\usepackage[]{amsmath}

\title{Unser erstes Dokument}
\author{Angelo Valentino Brade}
\date{\today}

\begin{document}

\maketitle
\tableofcontents

\section{Einführung}
Hier kommt die EInletiung. Ihre Überschift kommt automatisch in das Inhaltsverzeichnis.

\subsection*{Formeln}
\LaTeX{} ist auch ohne Formeln sehr nützlich und einfach zu verwenden. Grafiken, Tabellen, Querverweise aller Art, Literatur- und Stichwortverzeichnis sind kein Problem.
    Das ist mein erster Text \(f(x) = x^2 + x^3 + x^4 + \dots + x^n\).


\begin{eqnarray}
    3x_1 + 5x_2 - x_3 = 10
\end{eqnarray}



\end{document}